\documentclass{article}
\usepackage{amsmath,amssymb,amsthm,mathrsfs,graphicx,enumitem}
\usepackage{tikz, mathtools}
\usetikzlibrary{patterns}
\usepackage[margin=1in]{geometry}

\newtheorem*{problem}{Problem}
\newtheorem*{solution}{Solution}


\begin{document}

\section*{Problem Set Ch-1}

\begin{problem}{1.2}
    Prove that similar matrices have the same spectrum of eigenvalues.
\end{problem}

\begin{solution}
    
\end{solution}

\begin{problem}{1.7}
    Consider the matrix $\mathbf{A}$ that maps $\mathbb{R}^3$ into $\mathbb{R}^2$:

    \begin{center}
        \begin{math}
            \mathbf{A}= 
                \begin{bmatrix*}[r]
                        4 & -3 \\
                        -2 & 6 \\
                        4 & 6 \\ 
                \end{bmatrix*}
        \end{math}
    \end{center}

    \begin{enumerate}[label=\alph*.]
        \item Compute the SVD of $\mathbf{A}$ by hand and use it to find the null space of $\mathbf{A}$.
    \end{enumerate}

\end{problem}

\begin{solution}
    The SVD of $\mathbf{A}$ is given by $\mathbf{\Sigma} = \mathbf{U}^{H}\mathbf{A}\mathbf{V}$.
    Where $\mathbf{U}$ is the matrix of eigenvectors of 

    \begin{center}
        \begin{math}
            \mathbf{A}\mathbf{A}^{H}= 
                \begin{bmatrix*}[r]
                        25 & -26 & -2 \\
                        -26 & 40 & 28 \\
                        -2 & 28 & 52 \\ 
                \end{bmatrix*}
        \end{math}
    \end{center}

    and $\mathbf{V}$ is the matrix of eigenvectors of

    \begin{center}
        \begin{math}
            \mathbf{A}^{H}\mathbf{A}= 
                \begin{bmatrix*}[r]
                        28 & 0 \\
                        0 & 83 \\
                \end{bmatrix*}
        \end{math}.
    \end{center}

    The eigenvalues of $\mathbf{A}\mathbf{A}^{H}$ are given by

    \begin{center}
        \begin{math}
            \left\lvert \mathbf{A}\mathbf{A}^{H} - \lambda\mathbf{I} \right\rvert = 
                \begin{vmatrix*}[r]
                        25-\lambda & -26 & -2 \\
                        -26 & 40-\lambda & 28 \\
                        -2 & 28 & 52-\lambda \\ 
                \end{vmatrix*}
                = -\lambda^{3}+117\lambda^{2}-2916\lambda = -\lambda(\lambda - 36)(\lambda - 81)= 0.
        \end{math}
    \end{center}

    Thus, $\lambda = 0$, $\lambda = 36$, or $\lambda = 81$. Then $\sigma_1 = \sqrt{36} = 6$
    and $\sigma_1 = \sqrt{81} = 9$ and 

    \begin{center}
        \begin{math}
            \mathbf{\Sigma} = 
                \begin{bmatrix*}[r]
                    9 & 0 \\
                    0 & 6 \\
                    0 & 0 \\ 
                \end{bmatrix*}.
        \end{math}
    \end{center}

    When $\lambda = 0$,
    
    \begin{center}
        \begin{math}
            (\mathbf{A}\mathbf{A}^{H} - \lambda\mathbf{I})\mathbf{x} = 
                \begin{bmatrix*}[r]
                    25 & -26 & -2 \\
                    -26 & 40 & 28 \\
                    -2 & 28 & 52 \\ 
                \end{bmatrix*} \mathbf{x}
                =\mathbf{0}.
        \end{math}
    \end{center}

    Using row reduction we find $\mathbf{x} =     
    \begin{bmatrix*}[r]
        -2 \\
        -2 \\
        1 \\
    \end{bmatrix*}$ is an eigenvector. When $\lambda = 36$,
    
    \begin{center}
        \begin{math}
            (\mathbf{A}\mathbf{A}^{H} - \lambda\mathbf{I})\mathbf{x} = 
                \begin{bmatrix*}[r]
                    -11 & -26 & -2 \\
                    -26 & 4 & 28 \\
                    -2 & 28 & 16 \\ 
                \end{bmatrix*} \mathbf{x}
                =\mathbf{0}.
        \end{math}
    \end{center}

    Using row reduction we find $\mathbf{x} =     
    \begin{bmatrix*}[r]
        2 \\
        -1 \\
        2 \\
    \end{bmatrix*}$ is an eigenvector. When $\lambda = 81$,
    
    \begin{center}
        \begin{math}
            (\mathbf{A}\mathbf{A}^{H} - \lambda\mathbf{I})\mathbf{x} = 
                \begin{bmatrix*}[r]
                    -56 & -26 & -2 \\
                    -26 & -41 & 28 \\
                    -2 & 28 & -29 \\ 
                \end{bmatrix*} \mathbf{x}
                =\mathbf{0}.
        \end{math}
    \end{center}

    Using row reduction we find $\mathbf{x} =     
    \begin{bmatrix*}[r]
        -1 \\
        2 \\
        2 \\
    \end{bmatrix*}$ is an eigenvector. Then $\mathbf{U}$ is a matrix of
    the unit normalized eigenvectors. Thus, $\mathbf{U} =
    \begin{bmatrix*}[r]
        -\frac{1}{3} & \frac{2}{3} & \frac{2}{3} \\
        \frac{2}{3} & -\frac{1}{3} & \frac{2}{3} \\
        \frac{2}{3} & \frac{2}{3} & -\frac{1}{3} \\
    \end{bmatrix*}$ and $\mathbf{U}^{H} =     
    \begin{bmatrix*}[r]
        -\frac{1}{3} & \frac{2}{3} & \frac{2}{3} \\
        \frac{2}{3} & -\frac{1}{3} & \frac{2}{3} \\
        \frac{2}{3} & \frac{2}{3} & -\frac{1}{3} \\
    \end{bmatrix*}$.

    \bigskip

    The eigenvalues of $\mathbf{A}^{H}\mathbf{A}$ are given by

    \begin{center}
        \begin{math}
            \left\lvert \mathbf{A}^{H}\mathbf{A} - \lambda\mathbf{I} \right\rvert = 
                \begin{vmatrix*}[r]
                        28-\lambda & 0 \\
                        0 & 83-\lambda \\
                \end{vmatrix*}
                =(28 - \lambda)(83 - \lambda)= 0.
        \end{math}
    \end{center}

    Thus, $\lambda = 28$, or $\lambda = 83$. When $\lambda = 28$,

    \begin{center}
        \begin{math}
            (\mathbf{A}^{H}\mathbf{A} - \lambda\mathbf{I})\mathbf{x} = 
                \begin{bmatrix*}[r]
                        0 & 0 \\
                        0 & 55 \\
                \end{bmatrix*} \mathbf{x}
                =\mathbf{0}.
        \end{math}
    \end{center}

    Thus, $\mathbf{x} =                 
    \begin{bmatrix*}[r]
        1 \\
        0 \\
    \end{bmatrix*}$ is an eigenvector. When $\lambda = 83$,

    \begin{center}
        \begin{math}
            (\mathbf{A}^{H}\mathbf{A} - \lambda\mathbf{I})\mathbf{x} = 
                \begin{bmatrix*}[r]
                        -55 & 0 \\
                        0 & 0 \\
                \end{bmatrix*} \mathbf{x}
                =\mathbf{0}.
        \end{math}
    \end{center}

    Thus, $\mathbf{x} =                 
    \begin{bmatrix*}[r]
        0 \\
        1 \\
    \end{bmatrix*}$ is an eigenvector. Thus, $\mathbf{V} =     
    \begin{bmatrix*}[r]
        0 & 1 \\
        1 & 0 \\
    \end{bmatrix*}$.
    Then
    \begin{center}
        \begin{math}
            \mathbf{\Sigma} = \mathbf{U}^{H}\mathbf{A}\mathbf{V} = 
                \begin{bmatrix*}[r]
                    -\frac{1}{3} & \frac{2}{3} & \frac{2}{3} \\
                    \frac{2}{3} & -\frac{1}{3} & \frac{2}{3} \\
                    \frac{2}{3} & \frac{2}{3} & -\frac{1}{3} \\
                \end{bmatrix*}
                \begin{bmatrix*}[r]
                    4 & -3 \\
                    -2 & 6 \\
                    4 & 6 \\ 
                \end{bmatrix*}
                \begin{bmatrix*}[r]
                    0 & 1 \\
                    1 & 0 \\
                \end{bmatrix*}.
        \end{math}
    \end{center}

    As 
    \begin{math}
        \mathbf{V} =
            \begin{bmatrix*}[r]
                0 & 1 \\
                1 & 0 \\
            \end{bmatrix*},
    \end{math}
    $\mathbf{V}\mathbf{x} = \mathbf{0}$ has one solution namely $\mathbf{x} = \mathbf{0}$.
    Thus the null space of $\mathbf{A}$ is $\mathbf{0}$.
\end{solution}

\section*{Problem Set Ch-2}

\begin{problem}{2.5}
    Write a program to implement the reverse Cuthill-McKee algorithm for Symmetric matrices.
\end{problem}

\begin{solution}
    Solution can be found in CuthillMcKee.ipynb.
\end{solution}

\begin{problem}{2.6}
    Use the Cuthill-McKee algorithm to reorder the verticies in the circular
    graph layout of the bottlenose dolphin network in the example on page 49.
\end{problem}

\begin{solution}
    Solution can be found in CuthillMcKee.ipynb.
\end{solution}

\end{document}